\documentclass[addpoints]{exam}
\usepackage[utf8]{inputenc}
\usepackage[T1]{fontenc}
\usepackage[british]{babel}
\usepackage[defblank,oldenum]{paralist}
\usepackage[capitalize]{cleveref}
\usepackage{booktabs}

\usepackage[strict]{csquotes}
\usepackage[single]{acro}

\usepackage[natbib,style=alphabetic,maxbibnames=99]{biblatex}
\addbibresource{<package>.bib}

\usepackage{subfigure}

\usepackage[noend]{algpseudocode}
\usepackage{xparse}

\let\email\texttt

\usepackage{listings}
\lstset{%
  basicstyle=\footnotesize,
  numbers=left
}


\usepackage{color}
\usepackage{authblk}

%\printanswers

\title{<title>}
\date{<iso-date>}
\author{%
  <author>
}
\affil{%
  <affil>
}

\begin{document}
\maketitle
\thispagestyle{foot}

\section*{Instructions}

Carefully read the questions before you start answering them.
Note the time limit of the exam and plan your answers accordingly.
(The questions are \emph{not} sorted by difficulty.)
Only answer the question, do not write about subjects remotely related to the
question.

Make sure you follow these rules:
\begin{itemize}
  \item Write your answers on separate sheets, not on the exam paper.
  \item Only write on one side of the sheets.
  \item Start each question on a new sheet.
  \item Do not forget to \emph{motivate your answers.}
\end{itemize}

Make sure you write your answers clearly: if the examiner cannot read an 
answer, the answer will be awarded zero(!) points --- even if the answer is 
correct.

\begin{description}
  \item[Time] 5 hours.
  \item[Aids] Dictionary.
  \item[Questions] \numquestions
  \item[Maximum points] \numpoints
\end{description}

\subsection*{Grading}

Each question is worth three (3) points, representing how well you fulfil the 
intended learning outcomes covered by the question.
The following grading criteria applies:
\begin{description}
  \item[E] at least one point on each question.
  \item[D] closer to C than E.
  \item[C] at least two points on each question.
  \item[B] closer to A than C.
  \item[A] at least three points on each question.
\end{description}


\clearpage
\section*{Questions}

The questions are given below in uniformly random order.

\begin{questions}
  \input{questions-<date>.tex}
\end{questions}


\printbibliography
\end{document}

